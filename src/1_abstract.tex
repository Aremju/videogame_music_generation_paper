\begin{abstract}
Adaptive and generative music are two techniques commonly used in video games to create an immersive audio environment. Adaptive music modifies the game's soundtrack in response to the player's actions, while generative music uses algorithms to produce unique compositions in real-time. Both techniques provide game developers with powerful tools to craft engaging and memorable gaming experiences, and they are continually evolving with the latest technological advancements. Despite this progress, there are still technical limitations that hinder the commercial implementation of academic adaptive or generative music systems in real video games. In this paper, we introduce a range of frequently developed adaptive and generative music systems, explore their functionality, and evaluate their performance in terms of gameplay immersion. We also discuss the differences between these systems, their advantages and disadvantages, and potential future enhancements.
\end{abstract}
