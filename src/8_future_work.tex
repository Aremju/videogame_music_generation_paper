\section{Future Work}

In two of the three studies presented, the same video game was used to evaluate adaptive or generative music. In one study, the system’s output was integrated into two existing video games. To obtain meaningful results across different video game genres and types, further research is needed on the effects of adaptive and generative music. Additionally, the generative music system from the third study could be evaluated with more game iterations for each participant to determine if its musical variety affects players differently compared to the composed linear or adaptive score, which could feel repetitive after some iterations.

To address the runtime challenges of generative music models like presented in \cite{plut2022preglam}, it would be beneficial for future work to focus on optimizing these models and algorithms to enable real-time execution. Another approach would be to continue research into improving the hardware to compensate for the high resource demands of these models.

Further research should also aim at improving the consistency and reliability of adaptive and generative systems. Ensuring a certain quality of adaptive/generative music is challenging, as one cannot simply listen to the entire piece beforehand, unlike with linear composed music. Developing methods to assure quality without pre-listening is essential.

All studies relied on subjective reports from participants \cite{amaral2022adaptive}\cite{hutMcCormAms}\cite{plut2022preglam}\cite{lopez2023progressive}. Therefore, future work could explore more objective methods to measure the effects of adaptive and generative music on players. This might include physiological measures, in-game performance metrics, or other quantifiable data to complement subjective feedback and provide a more comprehensive understanding of the music's impact.
