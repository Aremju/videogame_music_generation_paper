\section{Introduction}
The aim of this paper is to summarise current developments and research on adaptive and generative music in video games. Therefore, different techniques and algorithms are described and their impact on the player is discussed. Furthermore, problems and challenges in the implementation of these systems are described and discussed.

- Beschreibung der Gliederung des Papers

- Diskussion, Ansichten, Meinungen/konkurrierende Ansätze darstellen


\subsection{Music in Video Games}
Music has an important effect in video games \cite{fu2015backgroundmusic}. Most video games use music to emphasize specific emotions and improve the tension and immersion. For this purpose linear music is used by the majority of video games \cite{prechtl2016adaptive}. Linear music is static and does not change in relation to the interaction of the player. But the video game itself is often not linear and reacts dynamically to player interactions. This can cause a conflict between the static audio and dynamic visual response of a video game, as the music may not fit perfectly to every state of the game.  

Due to hardware limitations, older video games such as Super Maio Bros from 1985 \cite{supermariobros1985} mostly use linear music \cite{plut2020generative}. However, advances in hardware have made it possible for current video games to implement customized music. This customized music can be grouped in adaptive and generative music \cite{plut2020generative}. In contrast to linear music, adaptive and generative music react to the player's interactions during the runtime of the video game \cite{plut2020generative}. This allows the music to be better adapted to the current game state.

\subsection{Adaptive Music}
Adaptive music, also referred to as interactive music, features a dynamic composition of different music fragments \cite{plut2020generative} \cite{hutMcCormAms} \cite{amaral2022adaptive}.
The composition can be organized using information about the game, such as variables or game states \cite{plut2020generative}. 
Furthermore, the composition of the music can also be changed on different levels such as tempo, pitch, volume \cite{plut2020generative} \cite{amaral2022adaptive}.

\subsection{Generative Music}
TODO
