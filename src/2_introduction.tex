\section{Introduction (Jonas)}
(lineare, adaptive, Generative Musik beschreiben, Beschreiben der Gliederung des Papers, was alles vorgestellt wird)
\begin{itemize}
    \item Thema beschreiben, Diskussion darstellen
    \item adaptive music
    \item generative music
    \item Ansichten, Meinungen/konkurrierende Ansätze darstellen
\end{itemize}


\subsection{Music in Video Games}
Music has an important effect in video games \cite{fu2015backgroundmusic}. Most video games use music to emphasize specific emotions and improve the tension and immersion. For this purpose linear music is used by the majority of video games \cite{prechtl2016adaptive}. Linear music is static and does not change in relation to the interaction of the player. But the video game itself is often not linear and reacts dynamically to player interactions. This can cause a conflict between the static audio and dynamic visual response of a video game, as the music may not fit perfectly to every state of the game.  

TODO: ältere videospiele technische begrenzungen, heute: technische entwicklungen: angepasste music möglich

\subsection{Adaptive Music}
TODO

\subsection{Generative Music}
TODO
