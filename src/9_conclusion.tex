\section{Conclusion}

In this paper, we have shown a set of adaptive and generative
music composition systems. There were systems like the AMS of Hutchings and McCormack
\cite{hutMcCormAms} or the Progressive-Adaptive Music Generator
of Lopez \cite{lopez2023progressive} that are hybrid, using 
adaptive and generative components in their systems. Next to both of those systems, we have shown approaches that utilize the Transformer architecture like the system of Amaral et al.
\cite{amaral2022adaptive} and the Multi-Track Music Machine by 
Ens and Pasquier \cite{ens2020mmm}. Some of them utilize an
emotion model like the AMS \cite{hutMcCormAms} and the approach
of Amaral et al. \cite{amaral2022adaptive}. Some of those
systems have been evaluated.

The evaluation studies on adaptive and generative music show how these technologies impact player experience. According to their results \cite{hutMcCormAms} \cite{plut2019music} \cite{plut2022preglam}, adaptive and generative music systems can increase the enjoyment, affect and immersion of video games by reacting to current game state and game variables. Generative music has the potential to enhance musical diversity and reduce the repetitive nature often associated with linear music. However, this has not yet been empirically evaluated in one of the studies presented.
Nevertheless, the results of the studies show that adaptive and generative music are overall less consistent and reliable in quality compared to hand-composed linear music. Additionally, generative music systems can be quite resource-intensive, which currently causes problems to integrate those systems into video games for real-time performance \cite{plut2022preglam}.
